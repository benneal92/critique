\documentclass[a4paper]{article}
\usepackage[T1]{fontenc}
\usepackage[utf8]{inputenc}
\usepackage{lmodern}
\usepackage[english]{babel}
\usepackage{csquotes}
\usepackage{enumitem}
\usepackage[notes,backend=bibtex]{biblatex-chicago}
\bibliography{C:/work/books/writing/psychiatry/references}
\usepackage{url} 

\begin{document}
\title{A Perspective}
\author{G. Benjamin Neal}
\date{\today}
\maketitle

\begin{abstract}
\end{abstract}

\section{Metaphysical Assumptions}
\subsection{Teleology}
At its core, teleology posits that organisms and systems exhibit goal-directed behaviors or ends. 
In moral philosophy, this view suggests that human flourishing or “the good” serves as an intrinsic endpoint of ethical action. 
Modern criticism often brands teleological accounts as anthropomorphic or scientifically dubious—yet this paper refrains from flatly denying goal-orientation. 
Instead it treats teleology as an implicit backdrop, inviting us to ask: does nature itself “aim” at anything, or do ends emerge only within interpretive frameworks?

\subsection{Naturalism}
Naturalism holds that all phenomena—mental, moral, or physical—are continuous with the natural world describable by empirical science. 
Within ethics, naturalism seeks to ground values in facts about human biology, psychology, or social evolution, avoiding appeals to the supernatural. 
Yet strict naturalism must confront the “is/ought” gap: deriving normative prescriptions purely from descriptive premises. 

\subsection{Materialism and Idealism}
Materialism asserts that matter and physical processes constitute the fundamental substance of reality. 
But a rigid materialism struggles to account for consciousness, intentionality, or value-laden experiences. 
“Room for idealism” introduces a qualified dual-aspect or emergentist stance: mental phenomena arise from physical substrates yet possess irreducible qualities. 
This hybrid view preserves a materialist ontology while acknowledging that ideas, values, and meanings are not mere epiphenomena—they actively shape both individual life and collective practices.

\section{Epistemological Arguments}
\subsection{The Role of Science}
Science is a systematic approach to understanding phenomena through observation, hypothesis, experimentation, and revision. 
At its core lies a cycle: we observe patterns, formulate explanatory models, test predictions, and refine or reject those models based on empirical evidence. 
This iterative process distinguishes science from mere opinion or anecdote, grounding claims in reproducible data and communal scrutiny. 
By emphasizing falsifiability and replication, the scientific method remains self-correcting, continually edging closer to models that reliably describe aspects of the world.

\subsection{The Role of Philosophy}
What is the use of Philosophy and its relation to Science is an ongoing contemporary debate. 
The question wether there are definite lines of demarcation or if borders are a gradient composed of a gradual shift between different disciplines is important to Epistemological concerns.

Facts alone are inert without a framework to interpret them. 
Raw data become significant only when woven into narratives that explain “why” and “how.” 
Integration involves three steps:
    Identifying patterns that call for explanation.
    Situating those patterns within broader theoretical or cultural contexts.
    Articulating a coherent story that links empirical findings to human concerns.

By exposing hidden assumptions, philosophy prevents science from mistaking provisional models for final truth.


\subsection{Open vs Closed Systems}
An open system exchanges matter, energy, or information with its environment, allowing adaptation to new inputs. 
A closed system, by contrast, operates in isolation, with fixed boundaries and no external influence. 
In scientific inquiry, treating a phenomenon as open invites complexity, feedback loops, and emergent behavior; treating it as closed fosters clarity and control but risks oversimplification. 
Choosing one perspective over the other shapes both experimental design and the kinds of explanations deemed acceptable.

\subsection{The problem of Reality}
Science is not static but a \textit{dynamic} and revisionary process—an ongoing attempt to correct perceived shortcomings.  
It is shaped not only by empirical findings but also by a developing scientific consciousness that seeks to \textit{interpret} and \textit{contextualize} those findings.

The notion that science functions as a mirror—merely reflecting the object as it is—fails to account for the subject’s role in shaping understanding.  
We do not stand outside the world we examine; we are always already participants.  
Facts are never encountered in isolation—they are filtered through the lens of our own cognition, culture, and expectation.  
Without self-awareness, interpretation becomes projection: the self mistaken for the world.

This tension may be characterized as \enquote{Science as Mirror} versus \enquote{Science as Medium}.  
Empirical findings are not raw truths; they are interpreted signals, refracted through conceptual frameworks, cultural assumptions, and personal biases.

\subsection{Science does not exist without Presuppositions}
Every scientific inquiry rests on background beliefs—about the reliability of the senses, the uniformity of nature, or the objectivity of measurement. 
These presuppositions are rarely stated, yet they constrain what counts as evidence and shape how anomalies are handled. 
Acknowledging them is not a weakness but a strength: it invites critical reflection on the limits and possibilities of our methods.

\subsection{Perspectivism}
Perspectivism holds that knowledge claims are always from a particular vantage point, never from a “view from nowhere.” 
Different cultural, historical, or disciplinary perspectives highlight distinct features of the same phenomenon. 
Rather than seeking a single, unified account, perspectivism encourages a plurality of models, each illuminating complementary dimensions of complex realities.

% TODO: Add more detailed discussion of objectivity concepts
The concept of objectivity, as discussed by Amartya Sen, involves both observation dependence and impersonality. 
In psychiatric practice, we see this tension manifest in concepts such as \enquote{interobserver variability} and the use of statistical measures like Kappa coefficients to assess diagnostic reliability within the DSM system of categorical classification.

\section{Situational Normality}
Before ethical reflection begins, we must examine the scaffolding upon which moral judgments are built. 
Normality is not a neutral backdrop—it is a dynamic construct shaped by context, consensus, and power. 
This section explores how normality emerges, how it functions epistemically, and how it quietly governs the boundaries of moral discourse.

Normality as a descriptive or sociological phenomenon—how it emerges, shifts, and operates across contexts.

Epistemic or psychological dimensions (e.g. how normality shapes perception, identity, or self-deception).
Critique of normativity itself, perhaps as a precursor to ethical reflection.

\subsubsection{Sociocultural}
\subsubsection{Historical}
\subsubsection{Functional}
\subsubsection{Power}

\section{The Ethical and Moral}
\subsection{Ammoralism}
Ammoralism denies that moral judgments have intrinsic authority, viewing them as contingent social constructs rather than universal truths.

\subsection{Emotivism}
Emotivism reduces moral statements to expressions of personal attitude or emotion—“stealing is wrong” becomes “I disapprove of stealing.”

\subsection{Normality and Moral Intuition}
Our intuitive sense of right and wrong often aligns with what a given community treats as normal. 
Disrupting norms can trigger moral anxiety or moral innovation, depending on the robustness of communal support for change.

\section{Selfhood}
\subsection{Nurture and Nature}
Identity arises from the interplay of biological endowment and social shaping. 
Nature provides the potentials; nurture scaffolds the actualization of those potentials within specific contexts.

Identity Informed by Biology
Genetic and neurological factors set constraints and affordances for personality, cognition, and emotion. 
Yet biology does not dictate destiny—cultural meaning-making and individual agency co-author our evolving sense of self.

Selfhood is a superset of both nurture and nature.

\subsection{Selfhood is Provisional}
The self is not a static entity but a perpetual becoming shaped by reflection and interaction. 
Every memory we choose to preserve, every role we perform, and every story we tell about ourselves stitches together the tapestry of identity.

Narrative Weaving: We construct our identity by selecting pivotal moments, interpreting them, and embedding them in personal myths.

Dialogical Formation: Internal dialogues—between critic and champion, doubt and hope—continuously revise our self-understanding.

Relational Co-creation: Each encounter with others acts as a mirror, revealing facets of ourselves and inviting reinterpretation.

This constructive process means selfhood is always provisional, open to new experiences and reframed meanings.

\subsection{Vitality and Self Actualization}

\section{Gender}
Gender names the layered ways in which cultures categorize, experience, and regulate bodies and selves. 
It differs from biological sex—those chromosomal, hormonal, and anatomical features—by denoting roles, expectations, and identities that vary across time and place. 
Understanding gender demands teasing apart (1) the lived sense of being a gendered subject, (2) the social scripts that channel behavior, and (3) the institutional norms that enforce boundaries.

\subsection{Social Construction and Normativity}
Gender norms emerge through collective practices—rituals, language, media tropes—that teach and police “appropriate” expressions of masculinity, femininity, or other genders. These norms function on multiple levels:

- Everyday interactions that reward conformity or stigmatize deviation
- Legal and policy frameworks that grant or restrict rights based on gender
- Cultural narratives that cast certain traits as inherently “male” or “female”

Because these norms shift, gender itself is a living scaffold: what counts as a respectable woman’s work or a proper man’s posture can change within a generation.

\subsection{Performance and Performativity}
Building on Judith Butler’s insight, gender is less a fixed attribute than an ongoing performance. 
Through repeated gestures—speech patterns, clothing choices, body language—we enact and reify gender roles. 
But with each performance lies the possibility of parody or rupture: subversive acts that expose norms as contingent, opening space for alternative expressions.

\subsection{Power, Resistance, and Fluidity}
Gender regimes distribute power unevenly, privileging some identities while marginalizing others. 
Resistance takes many forms: from collective protest to personal practices of self-definition. 
Increasingly, nonbinary, genderqueer, and two-spirit articulations challenge the binary scaffold, suggesting that gender flows along continua rather than between fixed poles.

\section{Medical Consciousness}
Medical consciousness is not merely a cognitive stance—it is a historical and ethical posture.  
It can reinforce dominant narratives or interrupt them.  
It can pathologize difference or illuminate suffering.  
It is shaped by the values we hold, the language we use, and the silences we permit.


\section{Contemporary Psychiatric Theory}


\subsection{Discrete Approach}

Many critiques have pointed out deep-seated problems with the DSM's diagnostic \enquote{pigeon-holing,} and many reforms, such as a diagnostic system based on dimensions of pathology, have been proposed. However, insurance systems require discrete codes, creating practical constraints on diagnostic innovation \cite{PubMedObjective}.

\begin{itemize}[leftmargin=2em]
    \item Genetics-based classifications
    \item Hereditary factors
    \item Neuroimaging correlates
    \item Cleanly separated from normalcy and from each other
    \item Categorical diagnostic systems
    \item Aristotelian category theory
\end{itemize}

\subsection{Dimensional Approach}

\begin{itemize}[leftmargin=2em]
    \item Continuum-based understanding
    \item Neuroimaging spectrum approaches
    \item Subclinical versions of psychiatric symptoms
    \item Influenced by Humean or Kantian philosophical frameworks
\end{itemize}

\subsection{Network Approach}

\begin{itemize}[leftmargin=2em]
    \item Interconnected symptom networks
    \item No central causal mechanism
    \item Neurobiological correlates
    \item Contextual sensitivity
    \item Computational modeling approaches
    \item Influenced by Whiteheadian or Deleuzian process philosophy
\end{itemize}

The fact that many patients meet criteria for multiple diagnoses simultaneously, or shift between diagnoses over time, further suggests the boundaries may be artificial constructs rather than natural kinds. This phenomenon challenges the discrete categorical approach and supports more fluid, dimensional or network-based conceptualizations.

\printbibliography

\end{document}