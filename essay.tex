\documentclass[a4paper]{article}
\usepackage[T1]{fontenc}
\usepackage[utf8]{inputenc}
\usepackage{lmodern}
\usepackage[english]{babel}
\usepackage{csquotes}

\usepackage[notes,backend=biber]{biblatex-chicago}
\bibliography{references}

\begin{document}
\title{Kant Stop Won't Stop: A Guide to the Critique of Pure Reason}
\author{G. Benjamin Neal}
\date{December 1, 2024}
\maketitle

\begin{abstract}
The aim of this paper is to provide an aid in any systematic exploration of Kant's Critique of Pure Reason. 
\end{abstract}

\section{Introduction}
In the Critique of Pure Reason, Immanuel Kant embarks to answer the question of how knowledge is possible.
Kant is not contented with the empiricist claim that all knowledge is derived from experience, nor with the rationalist claim that knowledge is derived from reason alone.
He sees a synthesis of both approaches as the key to understanding the nature of knowledge.
Kant's central insight is that knowledge is not a direct reflection of reality but a construction of the mind, shaped by its "inherent structures".
These structures, which Kant calls "categories", are the a priori concepts that the mind uses to organize sensory data into meaningful experiences.

\section{Vocabulary}
\subsection{A Priori}
Knowledge that is independent of experience, existing prior to any sensory input, like basic mathematical concepts.

\subsection{A Posteriori}
Knowledge that is derived from experience.

\subsection{Subject and Predicate}
The subject is what the sentence is about, and the predicate is what is being said about the subject.
Looking at the sentence, "The cat is sleepy."
The cat is the subject while "is sleepy" is the predicate.

\subsection{Analytic}
Judgments where the predicate is contained within the subject. 
Considering the statement "All bachelors are unmarried males".
The concept of a bachelors contains the concept of unmarried males. 

\subsection{Synthetic}
Judgments that extend our knowledge by adding something to the subject not contained in the definition of the subject.
For example, the statement "All bachelors enjoy their freedom" adds information about the bachelors that is not contained in the concept of bachelors itself.

\subsection{Synthetic a priori judgments}
Statements that are known to be true independently of experience but still provide new information about the world.

\subsection{Noumena}
The "thing-in-itself," the world as it exists independently of our perception.
Things-in-Themselves.

\subsection{Phenomena}
The world as we experience it. 
Things-as-They-Appear

\subsection{Intuition}
The immediate, raw sensory experience that forms the basis for our knowledge.
Can be thought as empirical intuition. 

\subsection{Apperception}
The act of self-awareness, the unifying principle that allows us to experience ourselves as a coherent subject.

\subsection{Concept}
A concept is a general idea or category that our mind uses to organize and interpret experiences. 
For example, the concept of "causality" helps us understand that one event can cause another. 
Concepts are the building blocks of our understanding and help us make sense of the sensory data we receive from the world.

\subsection{Principle}
A principle is a fundamental truth or law that governs how our concepts are applied and how knowledge is structured. 
Principles guide the use of concepts. 
For example, the "principle of sufficient reason" states that everything must have a reason or cause. 
This principle underlies our concept of causality and helps us apply it consistently to understand the world.

\subsection{Categories}
Categories are subset of concepts.
They are the most fundamental concepts of the understanding that structure our experiences (e.g., causality, substance).
Innate concepts like causality, substance, and time that the mind uses to organize sensory data into meaningful experiences.

These categories are not derived from experience but are necessary for the experience to be possible. 
Without these categories, our sensory input would be a chaotic stream of data with no coherence

\begin{table}[h!]
\centering
\begin{tabular}{|p{2cm}|p{10cm}|}
\hline
\textbf{Category} & \textbf{Description} \\
\hline
Quantity & Unity, Plurality, Totality \\
\hline
Quality & Reality, Negation, Limitation \\
\hline
Relation & Inherence and Subsistence (substance and accident), Causality and Dependence (cause and effect), Community (reciprocal action) \\
\hline
Modality & Possibility, Existence \\
\hline
\end{tabular}
\caption{Categories of Understanding}
\label{tab:categories}
\end{table}

\subsubsection{Quantity}
The category of quantity deals with the number of objects in a judgment \autocite{StanfordCategories}.
\begin{itemize}
    \item Unity - "All swans are white"
    \item Plurality - "Some swans are white"
    \item Totality - "Cygmund is white"
\end{itemize}

\subsubsection{Quality}
\begin{itemize}
    \item Reality - ""
    \item Negation - ""
    \item Limitation - ""
\end{itemize}

\subsubsection{Relation}
\begin{itemize}
    \item Unity - "All swans are white"
    \item Plurality - "Some swans are white"
    \item Totality - "Cygmund is white"
\end{itemize}

\subsubsection{Modality}
\begin{itemize}
    \item Possibility - ""
    \item Existence - ""
\end{itemize}

\subsection{}

\subsection{Things-in-Themselves}
Noumena, the world as it exists independently of our perception.

\subsection{Objective Reality}
This is often associated with the noumenal world (things-in-themselves) which we cannot know directly, but we assume it exists as the source of our sensory experiences.
Objective reality is about the existence of things independently of our perception.

\subsection{Objective Validity}
This is about the applicability and coherence of our concepts and judgments with respect to objects in our experience.
A concept or judgment has objective validity if it can be applied to the objects of our experience in a consistent and systematic way
Objective validity is about the coherence and reliability of our concepts and judgments within the realm of our experience.

\subsection{Schema}
A mental representation that connects our intuitions to the categories of understanding, allowing us to apply concepts to sensory data.

\subsection{Transcendental}

\subsection{Transcendental Aesthetic}
The part of the Critique that deals with the principles of sensibility, particularly space and time.

\subsection{Transcendental Idealism}
Knowledge is not a direct reflection of noumena, but rather a constructed experience shaped by our mind's categories.

\subsection{Transcendental Deduction}
The process by which Kant justifies the application of a priori concepts to experiences.
Kant's transcendental deduction is an attempt to show that the structures of our mind, which we bring to experience, are not arbitrary but are essential to how we perceive and understand the world. 
Without these structures, we wouldn't have coherent experiences or knowledge

\subsubsection{Subjective Deduction}
How do our minds take the raw data from our senses and turn it into a coherent experience? 
Emphasis is placed on the mental process.

\subsubsection{Objective Deduction}
Why are we justified in using concepts like causality or substance to understand and explain the world? 
Emphasis here is on the on the rational justification.


\subsection{Transcendental Logic}
The part of the Critique that deals with the principles of understanding and the application of the categories.


\section{Transcendental Doctrine of Elements}
\section{Transcendental Aesthetic}
\subsection{Space}
\subsection{Time}

\section{Transcendental Logic}
\subsection{Transcendental Analytic}
\subsubsection{Analytic of Concepts}
- Clue to the Discovery of all Pure Concepts of the Understanding
- Deduction of the Pure Concepts of the Understanding

\subsubsection{Analytic of Principles}
- Schematism of the Pure Concepts of the Understanding
- System of all Principles of the Pure Understanding
- Ground of the Distinction of all Objects in General into Phenomena and Noumena

\subsection{Transcendental Dialectic}
- On the Concepts and Principles of the Transcendental Dialectic
- The System of all Principles of Pure Reason
- The Ideal of Pure Reason

\section{Transcendental Doctrine of Method}

\subsection{The Discipline of Pure Reason}

\subsection{The Canon of Pure Reason}

\subsection{The Architectonic of Pure Reason}

\subsection{The History of Pure Reason}

\section{Conclusion}



\printbibliography

\end{document}
