\documentclass[a4paper]{article}
\usepackage[T1]{fontenc}
\usepackage[utf8]{inputenc}
\usepackage{lmodern}
\usepackage[english]{babel}
\usepackage{csquotes}

\usepackage[notes,backend=biber]{biblatex-chicago}
\bibliography{references}

\begin{document}
\title{A Map of The Critique of Pure Reason}
\author{G. Benjamin Neal}
\date{December 1, 2024}
\maketitle

\begin{abstract}
The aim of this paper is act as a signpost to the key concepts and ideas in Immanuel Kant's Critique of Pure Reason.
\end{abstract}

\section{Introduction}
The Critique of Pure Reason is most definitely a confusing read. 
Partly because of the new terminology Kant invented and partly because of the complexity of the ideas presented.
What does not help is that even Kant is perhaps inconsistent in his use of terms and ideas.
Likewise certain arguments are improperly developed and not fully explained.
That said the Critique of Pure Reason is a monumental work that has had a profound impact on philosophy and epistemology.

\section{Vocabulary}
This section will provide a brief explanation of some key terms and concepts that are sprinkled throughout the Critique.

\subsection{A Priori}
Knowledge that is independent of experience, existing prior to any sensory input, like basic mathematical concepts.

\subsection{A Posteriori}
Knowledge that is derived from experience.

\subsection{Subject and Predicate}
The subject is what the sentence is about, and the predicate is what is being said about the subject.
Looking at the sentence, "The cat is sleepy."
The cat is the subject while "is sleepy" is the predicate.

\subsection{Analytic}
Judgments where the predicate is contained within the subject. 
Considering the statement "All bachelors are unmarried males".
The concept of a bachelors contains the concept of unmarried males. 

\subsection{Synthetic}
Judgments that extend our knowledge by adding something to the subject not contained in the definition of the subject.
For example, the statement "All bachelors enjoy their freedom" adds information about the bachelors that is not contained in the concept of bachelors itself.

\subsection{Synthetic a priori judgments}
Statements that are known to be true independently of experience but still provide new information about the world.

\subsection{Noumena}
The "thing-in-itself," the world as it exists independently of our perception.

\subsection{Phenomena}
The world as we experience it. 

\subsection{Things-in-Themselves}
Noumena, the world as it exists independently of our perception.
Perhaps not directly knowable, but we can infer their existence.

\subsection{Things-as-They-Appear}
Phenomena, the world as we experience it.

\subsection{Sensibility}
The faculty of the mind that receives sensory input from the external world.
Sensibility is one of the two main faculties of the mind, alongside understanding.

\subsection{Immediate Knowledge}
Immediate knowledge is direct and does not rely on any intermediary processes or concepts. It is knowledge that we gain directly through intuition.
Our sensory perception of an object is immediate knowledge. \\

\textit{Example:} When you see a tree, you have immediate knowledge of its color, shape, and size through your direct perception.

\subsection{Mediate Knowledge}
Mediate knowledge is indirect and involves some form of mediation, such as conceptual thinking, reasoning, or inference. 
It is knowledge that we obtain through a process that involves interpreting or understanding something. \\

\textit{Example:}  When you reason that a tree must have roots because it is standing and growing, you are using mediate knowledge. 

\subsection{Intuition}
An Intuition is the immediate way in which we perceive objects through our senses before any conceptual thought or understanding is applied. 
It is the raw, direct perception of things as they appear to us.

\begin{enumerate}
    \item \textbf{Empirical Intuitions} Empirical Intuitions are contingent and dependent on sensory input. They provide the raw data for our experiences.
    \item \textbf{A Priori Intuitions} A Priori Intuitions are necessary and universal. They structure and organize the sensory data, making coherent experience possible.
\end{enumerate}


\subsection{Concept}
A concept is a general idea or category that our mind uses to organize and interpret experiences. 
For example, the concept of "causality" helps us understand that one event can cause another. 
Concepts are the building blocks of our understanding and help us make sense of the sensory data we receive from the world.

\subsection{Principle}
A principle is a fundamental truth or law that governs how our concepts are applied and how knowledge is structured. 
Principles guide the use of concepts. 
For example, the "principle of sufficient reason" states that everything must have a reason or cause. 
This principle underlies our concept of causality and helps us apply it consistently to understand the world.

\subsection{Judgment}
A judgment is a crucial activity of the mind that involves applying concepts to sensory data to form meaningful experiences
It involves bringing together a subject (the thing being considered) and a predicate (what is being said about the subject) into a coherent proposition.


\subsection{Logical Functions of Judgment}

\begin{table}[h!]
    \centering
    \begin{tabular}{|p{2cm}|p{10cm}|}
    \hline
    \textbf{Category} & \textbf{Description} \\
    \hline
    Quantity & Universal, Particular, Singular \\
    \hline
    Quality & Affirmative, Negative, Infinite \\
    \hline
    Relation & Categorical, Hypothetical, Disjunctive \\
    \hline
    Modality & Problematic, Assertoric, Apodeictic \\
    \hline
    \end{tabular}
    \caption{Logical Functions}
    \label{tab:categories}
    \end{table}

\subsection{Logical Functions of Judgment Explained}
\begin{enumerate}
    \item \textbf{Quantity:}
    \begin{itemize}
        \item \textbf{Universal:} Judgments that apply to all members of a category. \\
        \textit{Example:} All humans are mortal
        \item \textbf{Particular:} Judgments that apply to some members of a category. \\
        \textit{Example:} Some humans are tall
        \item \textbf{Singular:} Judgments that apply to a single member of a category. \\
        \textit{Example:} Socrates is wise
    \end{itemize}
    \item \textbf{Quality:}
    \begin{itemize}
        \item \textbf{Affirmative:} Judgments that affirm a property of a subject. \\
        \textit{Example:} The cat is black
        \item \textbf{Negative:} Judgments that deny a property of a subject. \\
        \textit{Example:} The cat is not black
        \item \textbf{Infinite:} Judgments that assert a non-property. \\
        \textit{Example:} The cat is non-white
    \end{itemize}
    \item \textbf{Relation:}
    \begin{itemize}
        \item \textbf{Categorical:} Simple subject-predicate judgments. \\
        \textit{Example:} The tree is tall.
        \item \textbf{Hypothetical:} Conditional judgments. \\
        \textit{Example:} If it rains, the ground will be wet.
        \item \textbf{Disjunctive:} Judgments that present alternatives. \\
        \textit{Example:} The light is either on or off.
    \end{itemize}
    \item \textbf{Modality:}
    \begin{itemize}
        \item \textbf{Problematic:} Judgments that express possibility. \\
        \textit{Example:} It may rain tomorrow.
        \item \textbf{Assertoric:} Judgments that express actuality. \\
        \textit{Example:} It is raining.
        \item \textbf{Apodeictic:} Judgments that express necessity. \\
        \textit{Example:} Two plus two must equal four.
    \end{itemize}
\end{enumerate}


\subsection{Categories}

\begin{table}[h!]
    \centering
    \begin{tabular}{|p{2cm}|p{10cm}|}
    \hline
    \textbf{Category} & \textbf{Description} \\
    \hline
    Quantity & Unity, Plurality, Totality \\
    \hline
    Quality & Reality, Negation, Limitation \\
    \hline
    Relation & Inherence and Subsistence (substance and accident), Causality and Dependence (cause and effect), Community (reciprocal action) \\
    \hline
    Modality & Possibility, Existence \\
    \hline
    \end{tabular}
    \caption{Categories of Understanding}
    \label{tab:categories}
\end{table}

Categories are subset of concepts.
They are the most fundamental concepts of the understanding that structure our experiences (e.g., causality, substance).
Innate concepts like causality, substance, and time that the mind uses to organize sensory data into meaningful experiences.

These categories are not derived from experience but are necessary for the experience to be possible. 
Without these categories, our sensory input would be a chaotic stream of data with no coherence. 

\begin{enumerate}
    \item \textbf{Quantity:}
    \begin{itemize}
        \item \textbf{Unity:} The concept of oneness or singularity. \\
        \textit{Example:} A single apple on a table.
        \item \textbf{Plurality:} The concept of manyness or multiplicity. \\
        \textit{Example:} Several apples in a basket.
        \item \textbf{Totality:} The concept of completeness or wholeness. \\
        \textit{Example:} All the apples in an orchard.
    \end{itemize}
    \item \textbf{Quality:}
    \begin{itemize}
        \item \textbf{Reality:} The concept of existence or actuality. \\
        \textit{Example:} The apple is red.
        \item \textbf{Negation:} The concept of non-existence or absence. \\
        \textit{Example:} The apple is not green.
        \item \textbf{Limitation:} The concept of restriction or boundary within existence. \\
        \textit{Example:} The apple is partly red and partly green.
    \end{itemize}
    \item \textbf{Relation:}
    \begin{itemize}
        \item \textbf{Inherence and Subsistence (substance and accident):} The concept of objects and their properties. \\
        \textit{Example:} The apple (substance) has a sweet taste (accident).
        \item \textbf{Causality and Dependence (cause and effect):} The concept of events and their causes. \\
        \textit{Example:} The apple fell because the branch broke.
        \item \textbf{Community (reciprocal interaction):} The concept of mutual influence or interaction between objects. \\
        \textit{Example:} Bees pollinate the apple blossoms, which leads to the production of apples.
    \end{itemize}
    \item \textbf{Modality:}
    \begin{itemize}
        \item \textbf{Possibility:} The concept of potentiality or what can be. \\
        \textit{Example:} It is possible for the apple tree to bear fruit next season.
        \item \textbf{Existence:} The concept of actuality or what is. \\
        \textit{Example:} The apple tree is currently bearing fruit.
        \item \textbf{Necessity:} The concept of inevitability or what must be. \\
        \textit{Example:} Given proper conditions, an apple seed must grow into an apple tree.
    \end{itemize}
\end{enumerate}

\subsection{Objective Reality}
This is often associated with the noumenal world (things-in-themselves) which we cannot know directly, but we assume it exists as the source of our sensory experiences.
Objective reality is about the existence of things independently of our perception.

\subsection{Objective Validity}
This is about the applicability and coherence of our concepts and judgments with respect to objects in our experience.
A concept or judgment has objective validity if it can be applied to the objects of our experience in a consistent and systematic way
Objective validity is about the coherence and reliability of our concepts and judgments within the realm of our experience.

\subsection{Manifold}
The manifold is the chaotic, unstructured flow of data that we receive. 
It's the initial input from our senses, but it hasn't yet been processed or organized by our understanding.

\begin{enumerate}
    \item \textbf{Empirical Manifold:} This is the sensory data we receive from the external world through our senses. It includes perceptions of objects and events in the environment. \\
    \textit{Example:} The colors, shapes, sounds, and textures we perceive around us constitute the empirical manifold.
    \item \textbf{Manifold of Inner Sense:} This refers to the data related to our inner experiences, such as thoughts, feelings, and mental states. This inner manifold is structured by the a priori intuition of time. \\
    \textit{Example:} Our awareness of our own thoughts, emotions, and mental processes.
\end{enumerate}

The \textit{Synthesis of Manifold} is the process by which the mind organizes and structures the manifold of sensory data into coherent experiences. \\

\textit{Example:} When we perceive a tree (external manifold), we also have an awareness of our perception of the tree (internal manifold), and these are synthesized into a unified experience.

\subsection{Synthesis}
The mind's role is to synthesize and organize this manifold of sensory data using its innate concepts and categories. 
This synthesis turns the raw data into coherent and meaningful experiences.

\subsection{Understanding}
The faculty of the mind that organizes and synthesizes sensory data into coherent experiences.
Mediate knowledge is associated with the understanding.

\subsection{Apperception}
The act of self-awareness, the unifying principle that allows us to experience ourselves as a coherent subject.

\subsection{Unity}
\begin{enumerate}
    \item \textbf{Unity of Apperception} This is the idea that our self-consciousness is unified. 
    In other words, when we say "I think," it implies that there is a single, unified self that is the subject of all our thoughts and experiences. 
    This unity is what allows us to combine different perceptions and thoughts into a coherent experience.
    \item \textbf{Synthetic Unity of Apperception} This refers to the process by which our mind synthesizes the manifold of sensory data into a unified experience. 
    The categories of understanding (such as causality, unity, etc.) play a crucial role in this process, helping us to organize and structure our experiences.
    \item \textbf{Unity of Experience} Kant argues that for us to have coherent and meaningful experiences, there must be a unity that binds our sensory data together. 
    This unity is provided by the mind's a priori concepts and categories, which structure our perceptions according to certain rules.
    \item \textbf{Unity of Space and Time} Space and time are the forms of our intuition, according to Kant. 
    They provide a unified framework within which we perceive and organize sensory data. 
    This spatial and temporal unity is essential for our experience of the external world.
\end{enumerate}

\subsection{Associations}
The association of representations is part of what Kant calls the "synthesis of reproduction in imagination." 
This synthesis helps us to recall past experiences and integrate them with current sensory data, creating a coherent flow of experience over time.


\begin{enumerate}
    \item \textbf{Contiguity} Representations that occur together in time or space tend to become associated with each other. \\
    \textit{Example:} If you always see a dog in your neighbor's yard, the sight of the yard may automatically bring the image of the dog to your mind.
    \item \textbf{Similarity} Representations that are similar to each other tend to become associated. \\
    \textit{Example:} Seeing one red apple might remind you of another red apple you saw previously.
    \item \textbf{Causality} Representations that are related causally tend to become associated. \\
    \textit{Example:} Hearing thunder might remind you of a lightning flash you saw earlier.
\end{enumerate}

These then can be broken down into subjective and objective associations.

\begin{enumerate}
    \item \textbf{Subjective Associations:}
    These associations are unique to the individual's personal experiences and memories. They might not be shared or understood by others. \\
    \textit{Example:} The smell of a certain perfume might remind you of a specific person because of your personal experiences.
    \item \textbf{Objective Associations:}
    These associations are based on general principles and are more likely to be shared and understood by others. \\
    \textit{Example:} The association between fire and heat is a common experience shared by most people.
\end{enumerate}

\subsection{Productive Imagination}
Productive imagination is the faculty of the mind that \textit{actively} generates and synthesizes new representations and ideas. 
It is creative and can produce novel concepts and images that are not directly derived from past experiences. \\

\textit{Example:} When we imagine a fictional creature like a unicorn, our productive imagination combines elements from different experiences (a horse and a horn) to create a new, novel representation

\subsection{Reproductive Imagination}
Reproductive imagination is the faculty of the mind that recalls and reproduces past experiences and representations. 
It is more \textit{passive} and relies on memory to bring back previous sensory data and experiences. \\

\textit{Example:} When we recognize a tree, our reproductive imagination recalls past experiences of trees to help us identify and understand the current sensory data.

\subsection{Schema}
A mental representation that connects our intuitions to the categories of understanding, allowing us to apply concepts to sensory data.

\subsection{Transcendental}

\subsection{Transcendental Object}
The transcendental object refers to the concept of an object that exists independently of our perception and understanding, 
but which serves as the ultimate reference point for all of our sensory experiences.

Like an idea rather than an actual physical object, the transcendental object is the source of our perceptions and the ground of our knowledge.

\subsection{Transcendental Aesthetic}
The part of the Critique that deals with the principles of sensibility, particularly space and time.

\subsection{Transcendental Idealism}
Knowledge is not a direct reflection of noumena, but rather a constructed experience shaped by our mind's categories.

\subsection{Transcendental Deduction}
The process by which Kant justifies the application of a priori concepts to experiences.
Kant's transcendental deduction is an attempt to show that the structures of our mind, which we bring to experience, are not arbitrary but are essential to how we perceive and understand the world. 
Without these structures, we wouldn't have coherent experiences or knowledge

\subsubsection{Subjective Deduction}
How do our minds take the raw data from our senses and turn it into a coherent experience? 
Emphasis is placed on the mental process.

\subsubsection{Objective Deduction}
Why are we justified in using concepts like causality or substance to understand and explain the world? 
Emphasis here is on the on the rational justification.

\subsection{Transcendental Logic}
The part of the Critique that deals with the principles of understanding and the application of the categories.

\section{Book Contents}
\subsection{Transcendental Doctrine of Elements}
\subsection{Transcendental Aesthetic}
\subsubsection{Space}
\subsubsection{Time}

\subsection{Transcendental Logic}
\subsubsection{Transcendental Analytic}
\subsubsection{Analytic of Concepts}
- Clue to the Discovery of all Pure Concepts of the Understanding
- Deduction of the Pure Concepts of the Understanding

\subsubsection{Analytic of Principles}
- Schematism of the Pure Concepts of the Understanding
- System of all Principles of the Pure Understanding
- Ground of the Distinction of all Objects in General into Phenomena and Noumena

\subsection{Transcendental Dialectic}
- On the Concepts and Principles of the Transcendental Dialectic
- The System of all Principles of Pure Reason
- The Ideal of Pure Reason

\section{Transcendental Doctrine of Method}

\subsection{The Discipline of Pure Reason}

\subsection{The Canon of Pure Reason}

\subsection{The Architectonic of Pure Reason}

\subsection{The History of Pure Reason}

\section{Conclusion}



\printbibliography
\end{document}
